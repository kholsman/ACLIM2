% Options for packages loaded elsewhere
\PassOptionsToPackage{unicode}{hyperref}
\PassOptionsToPackage{hyphens}{url}
%
\documentclass[
]{article}
\title{ACLIM2 ESR indices}
\author{K. Holsman}
\date{}

\usepackage{amsmath,amssymb}
\usepackage{lmodern}
\usepackage{iftex}
\ifPDFTeX
  \usepackage[T1]{fontenc}
  \usepackage[utf8]{inputenc}
  \usepackage{textcomp} % provide euro and other symbols
\else % if luatex or xetex
  \usepackage{unicode-math}
  \defaultfontfeatures{Scale=MatchLowercase}
  \defaultfontfeatures[\rmfamily]{Ligatures=TeX,Scale=1}
\fi
% Use upquote if available, for straight quotes in verbatim environments
\IfFileExists{upquote.sty}{\usepackage{upquote}}{}
\IfFileExists{microtype.sty}{% use microtype if available
  \usepackage[]{microtype}
  \UseMicrotypeSet[protrusion]{basicmath} % disable protrusion for tt fonts
}{}
\makeatletter
\@ifundefined{KOMAClassName}{% if non-KOMA class
  \IfFileExists{parskip.sty}{%
    \usepackage{parskip}
  }{% else
    \setlength{\parindent}{0pt}
    \setlength{\parskip}{6pt plus 2pt minus 1pt}}
}{% if KOMA class
  \KOMAoptions{parskip=half}}
\makeatother
\usepackage{xcolor}
\IfFileExists{xurl.sty}{\usepackage{xurl}}{} % add URL line breaks if available
\IfFileExists{bookmark.sty}{\usepackage{bookmark}}{\usepackage{hyperref}}
\hypersetup{
  pdftitle={ACLIM2 ESR indices},
  pdfauthor={K. Holsman},
  hidelinks,
  pdfcreator={LaTeX via pandoc}}
\urlstyle{same} % disable monospaced font for URLs
\usepackage[margin=1in]{geometry}
\usepackage{graphicx}
\makeatletter
\def\maxwidth{\ifdim\Gin@nat@width>\linewidth\linewidth\else\Gin@nat@width\fi}
\def\maxheight{\ifdim\Gin@nat@height>\textheight\textheight\else\Gin@nat@height\fi}
\makeatother
% Scale images if necessary, so that they will not overflow the page
% margins by default, and it is still possible to overwrite the defaults
% using explicit options in \includegraphics[width, height, ...]{}
\setkeys{Gin}{width=\maxwidth,height=\maxheight,keepaspectratio}
% Set default figure placement to htbp
\makeatletter
\def\fps@figure{htbp}
\makeatother
\setlength{\emergencystretch}{3em} % prevent overfull lines
\providecommand{\tightlist}{%
  \setlength{\itemsep}{0pt}\setlength{\parskip}{0pt}}
\setcounter{secnumdepth}{-\maxdimen} % remove section numbering
\usepackage{booktabs}
\usepackage{longtable}
\usepackage{array}
\usepackage{multirow}
\usepackage{wrapfig}
\usepackage{float}
\usepackage{colortbl}
\usepackage{pdflscape}
\usepackage{tabu}
\usepackage{threeparttable}
\usepackage{threeparttablex}
\usepackage[normalem]{ulem}
\usepackage{makecell}
\usepackage{xcolor}
\ifLuaTeX
  \usepackage{selnolig}  % disable illegal ligatures
\fi

\begin{document}
\maketitle

{
\setcounter{tocdepth}{3}
\tableofcontents
}
\begin{verbatim}
## Warning: package 'shiny' was built under R version 4.1.3
\end{verbatim}

\begin{verbatim}
## Warning: package 'ggnewscale' was built under R version 4.1.3
\end{verbatim}

\hypertarget{high-resolution-climate-change-projections-for-the-eastern-bering-sea}{%
\subsection{High resolution climate change projections for the Eastern
Bering
Sea}\label{high-resolution-climate-change-projections-for-the-eastern-bering-sea}}

Kirstin K. Holsman\emph{, Albert Hermann, Wei Cheng, Kelly Kearney,
Darren Pilcher, Kerim Aydin, Ivonne Ortiz}

*ESR contribution POC:
\href{mailto:kirstin.holsman@noaa.gov}{\nolinkurl{kirstin.holsman@noaa.gov}}

Alaska Fisheries Science Center, NOAA, 7600 Sand Point Way N.E., Bld. 4,
Seattle, Washington 98115

\textbf{Last Updated: October 2022}

\hypertarget{summary-statement}{%
\subsection{Summary statement:}\label{summary-statement}}

Projections of a high resolution oceanographic model for the Bering Sea
project wide spread warming across the region under high emission
scenarios with bottom temperatures exceeding historical ranges by
mid-century onward. Under both low and high emission scenarios the model
projects continued declines in ocean pH in bottom waters of the both the
North and Southern Bering Sea, especially during winter months but
declines are markedly larger and reach thresholds associated with
decreased survival of crab larvae (7.8 Long et al.~20XX) by mid-century,
and critical thresholds of high larval mortality by end of century (7.5
Long et al.~20xx). \emph{{[}ADD Caveat - ? Model results are not
observations and should be considered in the context of ongoing ocean
observations{]}.}

\hypertarget{status-and-trends}{%
\subsection{Status and trends:}\label{status-and-trends}}

Summer bottom temperatures in both the SEBS and NEBS are projected to
increase, with higher rates of warming associated with higher greenhouse
gas emission scenarios (SSP585). There is general agreement in all three
models with respect to trends in warming and across the three GCMS. For
the SEBS, estimates of end of century warming {[}2080-2100{]} range from
0.7 to 4.2 deg C and 2.8 to 6.1 deg C for SSP126 and SSP585,
respectively. In high emission scenarios, bottom temperatures for the
SEBS are projected to consistently to exceed the upper range of
historical temperatures predicted from the hindcast between 2050 and
2060.

\hypertarget{factors-influencing-observed-trends}{%
\subsection{Factors influencing observed
trends}\label{factors-influencing-observed-trends}}

\begin{itemize}
\tightlist
\item
  uptake of CO2 by oceans
\item
  uptake of atmospheric heat by oceans
\end{itemize}

Global increases in warming driven of approximately 1.1 deg. C over
postindustrial timeperiod is associated with significant warming of the
world's oceans and \ldots.{[} use details from IPCC AR6{]}

\hypertarget{implications}{%
\subsection{Implications:}\label{implications}}

\hypertarget{description-of-index}{%
\subsection{Description of index:}\label{description-of-index}}

We report trends in modeled bottom temperature and modeled bottom pH
from the 10 K Bering Sea regional Oceanographic model projected under
ACLIM2 CMIP6 simulations. The IEA operational hindcast and the ACLIM2
projections presented here are based on the K20P19 30-layer variant of
the Bering10K model that merges the biological module source code and
parameter updates from the K20 version (Kearney et al.~2020) with the
carbonate chemistry updates from the P19 (Pilcher et al.~2019) version.
See the (Bering 10K dataset
documentation){[}\url{https://zenodo.org/record/4586950/files/Bering10K_dataset_documentation.pdf}{]}
for more information and technical details. The two climate scenarios
and three global General Circulation Models (GCMs) were selected from
the Coupled Model Intercomparison Project phase 6 (CMIP6) and used to
force the boundary conditions of the regional model. See Hermann et
al.~2021, Cheng et al.~2021, Kearney et al.~2020, and Pilcher et
al.~2019 for details about model parameterization and Hollowed et
al.~2020 for details about the Alaska Climate Integrated Modeling
project and climate and GCM selection.

In support of the (Alaska Climate Integrated Modeling (ACLIM)
project){[}``www.fisheries.noaa.gov/alaska/ecosystems/alaska-climate-integrated-modeling-project''{]},
a number of different biophysical index timeseries were calculated based
on the Bering10K simulations and provide the primary means of linking
the physical and lower trophic level dynamics simulated by the Bering10K
long-term forecast simulations to the ACLIM suite of upper trophic level
and socioeconomic models; see Hollowed et al.~{[}2020{]} for further
details.The timeseries reported here are derived from the area-weighted
strata averages for Summer (months 7:9) and Winter for the Northern
Bering Sea (strata 70, 81, 82, 90) and Southern Bering sea. The
timerseries were bias corrected to the IEA operational hindcast using
historical runs for each GCM. More detail on this approach is available
by request.

The climate simulations presented here are dynamically downscaled from a
selection of the historical and shared socioeconomic pathway simulations
from the sixth phase of the Climate Model Intercomparison Project
(CMIP6){[}\url{https://www.wcrp-climate.org/wgcm-cmip/wgcm-cmip6}{]}.
Names reflect the parent model simulation (miroc = MIROC ES2L, cesm =
CESM2, gfdl = GFDL ESM4) and emissions scenario via Shared Socioeconomic
Pathways (SSPs) (ssp126 = SSP1-2.6, ssp585 = SSP5-8.5, historical =
Historical). SSP126 represents a lower atmospheric carbon emissions
scenario; SSP585 represents the high baseline emissions scenario. More
information on the SSPs and their use in climate projections is
available at O'Neil et al.~(2017){[}
\url{https://link.springer.com/article/10.1007/s10584-013-0905-2}{]}.

\hypertarget{literature-cited}{%
\subsection{Literature Cited}\label{literature-cited}}

Hermann et al., 2021 A.J. Hermann, K. Kearney, W. Cheng, D. Pilcher, K.
Aydin, K.K. Holsman, et al. Coupled modes of projected regional change
in the Bering Sea from a dynamically downscaling model under CMIP6
forcing Deep-Sea Res. II (2021), 10.1016/j.dsr2.2021.104974 194 104974

Cheng et al., 2021 W. Cheng, A.J. Hermann, A.B. Hollowed, K.K. Holsman,
K.A. Kearney, D.J. Pilcher, et al. Eastern Bering Sea shelf
environmental and lower trophic level responses to climate forcing:
results of dynamical downscaling from CMIP6 Deep-Sea Res. II, 193
(2021), Article 104975, 10.1016/j.dsr2.2021.104975

K. Kearney, A. Hermann, W. Cheng, I. Ortiz, and K. Aydin. A coupled
pelagicbenthic-sympagic biogeochemical model for the Bering Sea:
documentation and validation of the BESTNPZ model (v2019.08.23) within a
highresolution regional ocean model. Geoscientific Model Development, 13
(2):597--650, 2020. DOI: 10.5194/gmd13-597-2020. URL
\url{https://www.geosci-model-dev.net/13/597/2020/10}
\url{https://github.com/beringnpz/roms-bering-sea}

Pilcher, D. J., D. M. Naiman, J. N. Cross, A. J. Hermann, S. A.
Siedlecki, G. A. Gibson, and J. T. Mathis. Modeled Effect of Coastal
Biogeochemical Processes, Climate Variability, and Ocean Acidification
on Aragonite Saturation State in the Bering Sea. Frontiers in Marine
Science, 5(January):1--18, 2019. DOI: 10.3389/fmars.2018.00508 12
\url{https://github.com/beringnpz/} roms-bering-sea

Hollowed, K. K. Holsman, A. C. Haynie, A. J. Hermann, A. E. Punt, K. Y.
Aydin, J. N. Ianelli, S. Kasperski, W. Cheng, A. Faig, K. Kearney, J. C.
P. Reum, P. D. Spencer, I. Spies, W. J. Stockhausen, C. S. Szuwalski, G.
Whitehouse, and T. K. Wilderbuer. Integrated modeling to evaluate
climate change impacts on coupled social-ecological systems in Alaska.
Frontiers in Marine Science, 6(January):1--18, 2020. DOI:
10.3389/fmars.2019.00775

Projections of the high resolution Bering10K 30 layer model for the
Eastern Bering Sea. For more information see Hermman et al.~2021,
Kearney et al.~2020 and Pilcher et al.~2019. For more information about
climate scenarios selection see the Alaska Climate Integrate Modeling
project (ACLIM) website at
www.fisheries.noaa.gov/alaska/ecosystems/alaska-climate-integrated-modeling-project
and the Alaska NOAA Integrated Ecosystem Assessment program
www.integratedecosystemassessment.noaa.gov

\hypertarget{figures}{%
\subsection{Figures:}\label{figures}}

\begin{figure}
\centering
\includegraphics[width=1\textwidth,height=\textheight]{ESR_EBS/Figs/annualTS.png}
\caption{Bias corrected summer bottom temperature and winter pH for the
SEBS under low (SSP126) and high (SSP585) emission scenarios).}
\end{figure}

\begin{figure}
\centering
\includegraphics[width=1\textwidth,height=\textheight]{ESR_EBS/Figs/nonBC_weeklyProj_BT.png}
\caption{Bottom water temperature (degC) pojected under two climate
scenarios ( high carbon mitigation (ssp126); low carbon mitigation,
(SSP585) from three General Circulation Models (GCMs) dynamically
downscaled to a high resolution regional model (Bering10K 30 layer
ROMSNPZ model). 2022).}
\end{figure}

\begin{figure}
\centering
\includegraphics[width=1\textwidth,height=\textheight]{ESR_EBS/Figs/nonBC_weeklyProj_BT_N.png}
\caption{Bottom water temperature (degC) pojected under two climate
scenarios ( high carbon mitigation (ssp126); low carbon mitigation,
(SSP585) from three General Circulation Models (GCMs) dynamically
downscaled to a high resolution regional model (Bering10K 30 layer
ROMSNPZ model).}
\end{figure}

\begin{figure}
\centering
\includegraphics[width=1\textwidth,height=\textheight]{ESR_EBS/Figs/nonBC_weeklyProj_pH.png}
\caption{Bottom water pH pojected under two climate scenarios ( high
carbon mitigation (ssp126); low carbon mitigation, (SSP585) from three
General Circulation Models (GCMs) dynamically downscaled to a high
resolution regional model (Bering10K 30 layer ROMSNPZ model).}
\end{figure}

\begin{figure}
\centering
\includegraphics[width=1\textwidth,height=\textheight]{ESR_EBS/Figs/nonBC_weeklyProj_pH_N.png}
\caption{Bottom water pH pojected under two climate scenarios ( high
carbon mitigation (ssp126); low carbon mitigation, (SSP585) from three
General Circulation Models (GCMs) dynamically downscaled to a high
resolution regional model (Bering10K 30 layer ROMSNPZ model).}
\end{figure}

\end{document}
