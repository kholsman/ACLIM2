% Options for packages loaded elsewhere
\PassOptionsToPackage{unicode}{hyperref}
\PassOptionsToPackage{hyphens}{url}
%
\documentclass[
]{article}
\usepackage{amsmath,amssymb}
\usepackage{lmodern}
\usepackage{ifxetex,ifluatex}
\ifnum 0\ifxetex 1\fi\ifluatex 1\fi=0 % if pdftex
  \usepackage[T1]{fontenc}
  \usepackage[utf8]{inputenc}
  \usepackage{textcomp} % provide euro and other symbols
\else % if luatex or xetex
  \usepackage{unicode-math}
  \defaultfontfeatures{Scale=MatchLowercase}
  \defaultfontfeatures[\rmfamily]{Ligatures=TeX,Scale=1}
\fi
% Use upquote if available, for straight quotes in verbatim environments
\IfFileExists{upquote.sty}{\usepackage{upquote}}{}
\IfFileExists{microtype.sty}{% use microtype if available
  \usepackage[]{microtype}
  \UseMicrotypeSet[protrusion]{basicmath} % disable protrusion for tt fonts
}{}
\makeatletter
\@ifundefined{KOMAClassName}{% if non-KOMA class
  \IfFileExists{parskip.sty}{%
    \usepackage{parskip}
  }{% else
    \setlength{\parindent}{0pt}
    \setlength{\parskip}{6pt plus 2pt minus 1pt}}
}{% if KOMA class
  \KOMAoptions{parskip=half}}
\makeatother
\usepackage{xcolor}
\IfFileExists{xurl.sty}{\usepackage{xurl}}{} % add URL line breaks if available
\IfFileExists{bookmark.sty}{\usepackage{bookmark}}{\usepackage{hyperref}}
\hypersetup{
  pdftitle={ACLIM2 CMIP6 ROMSNPZ Indices quick start guide},
  pdfauthor={K. Holsman},
  hidelinks,
  pdfcreator={LaTeX via pandoc}}
\urlstyle{same} % disable monospaced font for URLs
\usepackage[margin=1in]{geometry}
\usepackage{color}
\usepackage{fancyvrb}
\newcommand{\VerbBar}{|}
\newcommand{\VERB}{\Verb[commandchars=\\\{\}]}
\DefineVerbatimEnvironment{Highlighting}{Verbatim}{commandchars=\\\{\}}
% Add ',fontsize=\small' for more characters per line
\usepackage{framed}
\definecolor{shadecolor}{RGB}{248,248,248}
\newenvironment{Shaded}{\begin{snugshade}}{\end{snugshade}}
\newcommand{\AlertTok}[1]{\textcolor[rgb]{0.94,0.16,0.16}{#1}}
\newcommand{\AnnotationTok}[1]{\textcolor[rgb]{0.56,0.35,0.01}{\textbf{\textit{#1}}}}
\newcommand{\AttributeTok}[1]{\textcolor[rgb]{0.77,0.63,0.00}{#1}}
\newcommand{\BaseNTok}[1]{\textcolor[rgb]{0.00,0.00,0.81}{#1}}
\newcommand{\BuiltInTok}[1]{#1}
\newcommand{\CharTok}[1]{\textcolor[rgb]{0.31,0.60,0.02}{#1}}
\newcommand{\CommentTok}[1]{\textcolor[rgb]{0.56,0.35,0.01}{\textit{#1}}}
\newcommand{\CommentVarTok}[1]{\textcolor[rgb]{0.56,0.35,0.01}{\textbf{\textit{#1}}}}
\newcommand{\ConstantTok}[1]{\textcolor[rgb]{0.00,0.00,0.00}{#1}}
\newcommand{\ControlFlowTok}[1]{\textcolor[rgb]{0.13,0.29,0.53}{\textbf{#1}}}
\newcommand{\DataTypeTok}[1]{\textcolor[rgb]{0.13,0.29,0.53}{#1}}
\newcommand{\DecValTok}[1]{\textcolor[rgb]{0.00,0.00,0.81}{#1}}
\newcommand{\DocumentationTok}[1]{\textcolor[rgb]{0.56,0.35,0.01}{\textbf{\textit{#1}}}}
\newcommand{\ErrorTok}[1]{\textcolor[rgb]{0.64,0.00,0.00}{\textbf{#1}}}
\newcommand{\ExtensionTok}[1]{#1}
\newcommand{\FloatTok}[1]{\textcolor[rgb]{0.00,0.00,0.81}{#1}}
\newcommand{\FunctionTok}[1]{\textcolor[rgb]{0.00,0.00,0.00}{#1}}
\newcommand{\ImportTok}[1]{#1}
\newcommand{\InformationTok}[1]{\textcolor[rgb]{0.56,0.35,0.01}{\textbf{\textit{#1}}}}
\newcommand{\KeywordTok}[1]{\textcolor[rgb]{0.13,0.29,0.53}{\textbf{#1}}}
\newcommand{\NormalTok}[1]{#1}
\newcommand{\OperatorTok}[1]{\textcolor[rgb]{0.81,0.36,0.00}{\textbf{#1}}}
\newcommand{\OtherTok}[1]{\textcolor[rgb]{0.56,0.35,0.01}{#1}}
\newcommand{\PreprocessorTok}[1]{\textcolor[rgb]{0.56,0.35,0.01}{\textit{#1}}}
\newcommand{\RegionMarkerTok}[1]{#1}
\newcommand{\SpecialCharTok}[1]{\textcolor[rgb]{0.00,0.00,0.00}{#1}}
\newcommand{\SpecialStringTok}[1]{\textcolor[rgb]{0.31,0.60,0.02}{#1}}
\newcommand{\StringTok}[1]{\textcolor[rgb]{0.31,0.60,0.02}{#1}}
\newcommand{\VariableTok}[1]{\textcolor[rgb]{0.00,0.00,0.00}{#1}}
\newcommand{\VerbatimStringTok}[1]{\textcolor[rgb]{0.31,0.60,0.02}{#1}}
\newcommand{\WarningTok}[1]{\textcolor[rgb]{0.56,0.35,0.01}{\textbf{\textit{#1}}}}
\usepackage{graphicx}
\makeatletter
\def\maxwidth{\ifdim\Gin@nat@width>\linewidth\linewidth\else\Gin@nat@width\fi}
\def\maxheight{\ifdim\Gin@nat@height>\textheight\textheight\else\Gin@nat@height\fi}
\makeatother
% Scale images if necessary, so that they will not overflow the page
% margins by default, and it is still possible to overwrite the defaults
% using explicit options in \includegraphics[width, height, ...]{}
\setkeys{Gin}{width=\maxwidth,height=\maxheight,keepaspectratio}
% Set default figure placement to htbp
\makeatletter
\def\fps@figure{htbp}
\makeatother
\setlength{\emergencystretch}{3em} % prevent overfull lines
\providecommand{\tightlist}{%
  \setlength{\itemsep}{0pt}\setlength{\parskip}{0pt}}
\setcounter{secnumdepth}{-\maxdimen} % remove section numbering
\usepackage{booktabs}
\usepackage{longtable}
\usepackage{array}
\usepackage{multirow}
\usepackage{wrapfig}
\usepackage{float}
\usepackage{colortbl}
\usepackage{pdflscape}
\usepackage{tabu}
\usepackage{threeparttable}
\usepackage{threeparttablex}
\usepackage[normalem]{ulem}
\usepackage{makecell}
\usepackage{xcolor}
\ifluatex
  \usepackage{selnolig}  % disable illegal ligatures
\fi

\title{ACLIM2 CMIP6 ROMSNPZ Indices quick start guide}
\author{K. Holsman}
\date{}

\begin{document}
\maketitle

{
\setcounter{tocdepth}{3}
\tableofcontents
}
\hypertarget{download-the-aclim2-repo-data}{%
\section{Download the ACLIM2 repo \&
data}\label{download-the-aclim2-repo-data}}

\hypertarget{clone-the-aclim2-repo}{%
\subsection{Clone the ACLIM2 repo}\label{clone-the-aclim2-repo}}

To run this tutorial first clone the ACLIM2 repository to your local
drive:

\hypertarget{option-1-use-r}{%
\subsubsection{Option 1: Use R}\label{option-1-use-r}}

This set of commands, run within R, downloads the ACLIM2 repository and
unpacks it, with the ACLIM2 directory structrue being located in the
specified \texttt{download\_path}. This also performs the folder
renaming mentioned in Option 2.

\begin{Shaded}
\begin{Highlighting}[]
    \CommentTok{\# Specify the download directory}
\NormalTok{    main\_nm       }\OtherTok{\textless{}{-}} \StringTok{"ACLIM2"}

    \CommentTok{\# Note: Edit download\_path for preference}
\NormalTok{    download\_path }\OtherTok{\textless{}{-}}  \FunctionTok{path.expand}\NormalTok{(}\StringTok{"\textasciitilde{}"}\NormalTok{)}
\NormalTok{    dest\_fldr     }\OtherTok{\textless{}{-}} \FunctionTok{file.path}\NormalTok{(download\_path,main\_nm)}
    
\NormalTok{    url           }\OtherTok{\textless{}{-}} \StringTok{"https://github.com/kholsman/ACLIM2/archive/main.zip"}
\NormalTok{    dest\_file     }\OtherTok{\textless{}{-}} \FunctionTok{file.path}\NormalTok{(download\_path,}\FunctionTok{paste0}\NormalTok{(main\_nm,}\StringTok{".zip"}\NormalTok{))}
    \FunctionTok{download.file}\NormalTok{(}\AttributeTok{url=}\NormalTok{url, }\AttributeTok{destfile=}\NormalTok{dest\_file)}
    
    \CommentTok{\# unzip the .zip file (manually unzip if this doesn\textquotesingle{}t work)}
    \FunctionTok{setwd}\NormalTok{(download\_path)}
    \FunctionTok{unzip}\NormalTok{ (dest\_file, }\AttributeTok{exdir =}\NormalTok{ download\_path,}\AttributeTok{overwrite =}\NormalTok{ T)}
    
    \CommentTok{\#rename the unzipped folder from ACLIM2{-}main to ACLIM2}
    \FunctionTok{file.rename}\NormalTok{(}\FunctionTok{paste0}\NormalTok{(main\_nm,}\StringTok{"{-}main"}\NormalTok{), main\_nm)}
    \FunctionTok{setwd}\NormalTok{(main\_nm)}
\end{Highlighting}
\end{Shaded}

\hypertarget{option-2-download-the-zipped-repo}{%
\subsubsection{Option 2: Download the zipped
repo}\label{option-2-download-the-zipped-repo}}

Download the full zip archive directly from the
\href{https://github.com/kholsman/ACLIM2}{\textbf{ACLIM2 Repo}} using
this link:
\href{https://github.com/kholsman/ACLIM2/archive/main.zip}{\textbf{https://github.com/kholsman/ACLIM2/archive/main.zip}},
and unzip its contents while preserving directory structure.

\textbf{Important!}* If downloading from zip, please \textbf{rename the
root folder} from \texttt{ACLIM2-main} (in the zipfile) to
\texttt{ACLIM2} (name used in cloned copies) after unzipping, for
consistency in the following examples.

Your final folder structure should look like this:

\includegraphics[width=1\textwidth,height=\textheight]{Figs/ACLIM_dir.png}

\hypertarget{option-3-use-git-commandline}{%
\subsubsection{Option 3: Use git
commandline}\label{option-3-use-git-commandline}}

If you have git installed and can work with it, this is the preferred
method as it preserves all directory structure and can aid in future
updating. Use this from a \textbf{terminal command line, not in R}, to
clone the full ACLIM2 directory and sub-directories:

\begin{Shaded}
\begin{Highlighting}[]
    \FunctionTok{git}\NormalTok{ clone https://github.com/kholsman/ACLIM2.git}
\end{Highlighting}
\end{Shaded}

\begin{center}\rule{0.5\linewidth}{0.5pt}\end{center}

\hypertarget{get-the-data}{%
\subsection{Get the data}\label{get-the-data}}

Go to the google drive and download the zipped file with the R data
`2022\_03\_07\_Rdata.zip':

\href{https://drive.google.com/drive/folders/11BQEfNEl9vvrN-V0LgS67XS4aLE9pNzz}{00\_ACLIM\_shared
\textgreater{} 02\_Data \textgreater{} Newest\_Data(use this)
\textgreater{} 2022\_03\_07\_Rdata.zip}

Move the Zipped folder to your local folder `ACLIM2/Data/in' and unzip.
The final folder structure should look like:

\includegraphics[width=0.5\textwidth,height=\textheight]{Figs/DATA_dir.png}

\hypertarget{set-up-the-workspace}{%
\subsection{Set up the Workspace}\label{set-up-the-workspace}}

Open R() and used `setwd()' to navigate to the root ACLIM2 folder (.e.g,
\textasciitilde/mydocuments/ACLIM2)

\begin{Shaded}
\begin{Highlighting}[]
    \CommentTok{\# set the workspace to your local ACLIM2 folder}
    \CommentTok{\# e.g., "/Users/kholsman/Documents/GitHub/ACLIM2"}
    \CommentTok{\# setwd( path.expand("\textasciitilde{}/Documents/GitHub/ACLIM2") )}
   
    \CommentTok{\# {-}{-}{-}{-}{-}{-}{-}{-}{-}{-}{-}{-}{-}{-}{-}{-}{-}{-}{-}{-}{-}{-}{-}{-}{-}{-}{-}{-}{-}{-}{-}{-}{-}{-}{-}{-}{-}{-}}
    \CommentTok{\# SETUP WORKSPACE}
\NormalTok{    tmstp  }\OtherTok{\textless{}{-}} \FunctionTok{format}\NormalTok{(}\FunctionTok{Sys.time}\NormalTok{(), }\StringTok{"\%Y\_\%m\_\%d"}\NormalTok{)}
\NormalTok{    main   }\OtherTok{\textless{}{-}} \FunctionTok{getwd}\NormalTok{()  }\CommentTok{\#"\textasciitilde{}/GitHub\_new/ACLIM2"}
    
    \CommentTok{\# loads packages, data, setup, etc.}
    \FunctionTok{suppressWarnings}\NormalTok{(}\FunctionTok{source}\NormalTok{(}\StringTok{"R/make.R"}\NormalTok{))}
\end{Highlighting}
\end{Shaded}

\begin{verbatim}
## ------------------------------
## ALIM2/R/setup.R settings 
## ------------------------------
## main:                : /Users/kholsman/Documents/GitHub/ACLIM2 
## data_path            : /Users/kholsman/Documents/GitHub/ACLIM2/Data/in/2022_03_07/roms_for_public 
## Rdata_path           : /Users/kholsman/Documents/GitHub/ACLIM2/Data/in/2022_03_07_Rdata/roms_for_public 
## redownload_level3_mox: FALSE 
## update.figs          : FALSE 
## load_gis             : FALSE 
## update.outputs       : TRUE 
## update.figs          : FALSE 
## dpiIN                : 150 
## update.figs          : FALSE 
## ------------------------------
## ------------------------------
## 
## The following datasets are public, please cite as Hermann et al. 2019 (v.H16) and Kearney et al. 2020 (v.K20) :
## B10K-H16_CMIP5_CESM_BIO_rcp85 
## B10K-H16_CMIP5_CESM_rcp45 
## B10K-H16_CMIP5_CESM_rcp85 
## B10K-H16_CMIP5_GFDL_BIO_rcp85 
## B10K-H16_CMIP5_GFDL_rcp45 
## B10K-H16_CMIP5_GFDL_rcp85 
## B10K-H16_CMIP5_MIROC_rcp45 
## B10K-H16_CMIP5_MIROC_rcp85 
## B10K-H16_CORECFS 
## B10K-K20_CORECFS 
## 
## The following datasets are still under embargo, please do not share outside of ACLIM:
## B10K-K20P19_CMIP6_cesm_historical 
## B10K-K20P19_CMIP6_cesm_ssp126 
## B10K-K20P19_CMIP6_cesm_ssp585 
## B10K-K20P19_CMIP6_gfdl_historical 
## B10K-K20P19_CMIP6_gfdl_ssp126 
## B10K-K20P19_CMIP6_gfdl_ssp585 
## B10K-K20P19_CMIP6_miroc_historical 
## B10K-K20P19_CMIP6_miroc_ssp126 
## B10K-K20P19_CMIP6_miroc_ssp585
\end{verbatim}

\begin{center}\rule{0.5\linewidth}{0.5pt}\end{center}

\hypertarget{read-this-before-you-start}{%
\section{Read this before you start}\label{read-this-before-you-start}}

\hypertarget{overview}{%
\subsection{Overview}\label{overview}}

The \href{https://github.com/kholsman/ACLIM2}{\textbf{ACLIM2 github
repository}} contains R code and Rdata files for working with
netcdf-format data generated from the
\href{https://beringnpz.github.io/roms-bering-sea}{\textbf{downscaled
ROMSNPZ modeling}} of the ROMSNPZ Bering Sea Ocean Modeling team; Drs.
Hermann, Cheng, Kearney, Pilcher,Ortiz, and Aydin. The code and R
resources described in this tutorial are maintained by
\href{mailto:kirstin.holsman@noaa.gov}{Kirstin Holsman} as part of
NOAA's
\href{https://www.fisheries.noaa.gov/alaska/ecosystems/alaska-climate-integrated-modeling-project}{\textbf{ACLIM
project}} for the Bering Sea. \emph{See
\href{https://www.frontiersin.org/articles/10.3389/fmars.2019.00775/full}{Hollowed
et al.~2020} for more information about the ACLIM project.}

\begin{center}\rule{0.5\linewidth}{0.5pt}\end{center}

This document provides an overview of accessing, plotting, and creating
bias corrected indices for ACLIM2 based on CMIP6 (embargoed for ACLIM2
users until 2023) and CMIP5 (publicly available) simulations. This guide
assumes analyses will take place in R() and that users have access to
the data folder within the ACLIM2 shared drive. For more information
also see the full tutorial (``GettingStarted\_Bering10K\_ROMSNPZ''
available at the bottom of
\href{https://github.com/kholsman/ACLIM2}{\textbf{this repo page}}.

\textbf{Important!} A few key things to know before getting started are
detailed below. Please review this information before getting started.

\hypertarget{key-romsnpz-versions}{%
\subsection{KEY ROMSNPZ versions}\label{key-romsnpz-versions}}

\textbf{Important!} ACLIM1 CMIP5 and ACLIM2 CMIP5 and CMIP6 datasets use
different base models.

There are two versions of the ROMSNPZ model:

\begin{enumerate}
\def\labelenumi{\arabic{enumi}.}
\tightlist
\item
  ACLIM1 an older 10-depth layer model used for CMIP5 (``H-16'')
\item
  ACLIM2 a new 30-depth layer model used for CMIP6 (``K20'' or
  ``K20P19'')
\end{enumerate}

The models are not directly comparable, therefore the projections should
be bias corrected and recentered to baselines of hindcasts of each model
(forced by ``observed'' climate conditions). i.e.~CMIP5 and CMIP6 have
corresponding hindcasts:

\begin{enumerate}
\def\labelenumi{\arabic{enumi}.}
\tightlist
\item
  Hindcast for CMIP5 ``H19'' --\textgreater{} H16\_CORECFS
\item
  Hindcast for CMIP5 ``K20P19'' --\textgreater{} H16\_CORECFS
\item
  Hindcast for CMIP6 ``K20P19'' --\textgreater{} K20\_CORECFS
\end{enumerate}

In addition for CMIP6 ``historical'' runs are available for bias
correcting. We will use those below.

For a list of the available simulations for ACLIM enter the following in
R():

\begin{verbatim}
# list of the climate scenarios
data.frame(sim_list)
\end{verbatim}

\hypertarget{key-data-outputs}{%
\subsection{KEY Data outputs}\label{key-data-outputs}}

\textbf{Important!} There are 2 types of post-processed data available
for use in ACLIM.

The ROMSNPZ team has developed a process to provide standardized
post-processed outputs from the large (and non-intuitive) ROMSNPZ grid.
These have been characterized as:

\begin{enumerate}
\def\labelenumi{\arabic{enumi}.}
\tightlist
\item
  Level 1 (original ROMSNPZ U,V, grid, not rotated or corrected)\\
\item
  Level 2 (lat long bi-weekly high res versions, shouldn't be needed and
  are difficult to work with)\\
\item
  \textbf{Level 3 indices (depth corrected and area weighted means for
  each model variable; i.e., what we will mostly use) }

  \begin{enumerate}
  \def\labelenumii{\alph{enumii}.}
  \tightlist
  \item
    ``ACLIMsurveyrep\_'': groundifsh survey replicated (replicated in
    space and time)
  \item
    ``ACLIMregion\_'': weekly strata based averages
  \end{enumerate}
\end{enumerate}

To get more information about each of these level 3 datasets enter this
in R:

\begin{Shaded}
\begin{Highlighting}[]
    \CommentTok{\# Metadata for Weekly ("ACLIMregion\_...") indices}
    \FunctionTok{head}\NormalTok{(all\_info1)}
\end{Highlighting}
\end{Shaded}

\begin{verbatim}
##                            name                    Type B10KVersion  CMIP  GCM
## 1 B10K-H16_CMIP5_CESM_BIO_rcp85 Weekly regional indices         H16 CMIP5 CESM
## 2     B10K-H16_CMIP5_CESM_rcp45 Weekly regional indices         H16 CMIP5 CESM
## 3     B10K-H16_CMIP5_CESM_rcp85 Weekly regional indices         H16 CMIP5 CESM
## 4 B10K-H16_CMIP5_GFDL_BIO_rcp85 Weekly regional indices         H16 CMIP5 GFDL
## 5     B10K-H16_CMIP5_GFDL_rcp45 Weekly regional indices         H16 CMIP5 GFDL
## 6     B10K-H16_CMIP5_GFDL_rcp85 Weekly regional indices         H16 CMIP5 GFDL
##     BIO Carbon_scenario               Start                 End nvars
## 1  TRUE           rcp85 2006-01-22 12:00:00 2099-12-27 12:00:00    59
## 2 FALSE           rcp45 2006-01-22 12:00:00 2081-02-16 12:00:00    59
## 3 FALSE           rcp85 2006-01-22 12:00:00 2099-12-27 12:00:00    59
## 4  TRUE           rcp85 2006-01-22 12:00:00 2099-12-27 12:00:00    59
## 5 FALSE           rcp45 2006-01-22 12:00:00 2099-12-27 12:00:00    59
## 6 FALSE           rcp85 2006-01-22 12:00:00 2099-12-27 12:00:00    59
\end{verbatim}

\begin{Shaded}
\begin{Highlighting}[]
    \CommentTok{\# Metadata for Weekly ("ACLIMsurveyrep\_...") indices}
    \FunctionTok{head}\NormalTok{(all\_info2)}
\end{Highlighting}
\end{Shaded}

\begin{verbatim}
##                            name              Type B10KVersion  CMIP  GCM   BIO
## 1 B10K-H16_CMIP5_CESM_BIO_rcp85 Survey replicated         H16 CMIP5 CESM  TRUE
## 2     B10K-H16_CMIP5_CESM_rcp45 Survey replicated         H16 CMIP5 CESM FALSE
## 3     B10K-H16_CMIP5_CESM_rcp85 Survey replicated         H16 CMIP5 CESM FALSE
## 4 B10K-H16_CMIP5_GFDL_BIO_rcp85 Survey replicated         H16 CMIP5 GFDL  TRUE
## 5     B10K-H16_CMIP5_GFDL_rcp45 Survey replicated         H16 CMIP5 GFDL FALSE
## 6     B10K-H16_CMIP5_GFDL_rcp85 Survey replicated         H16 CMIP5 GFDL FALSE
##   Carbon_scenario Start  End nvars
## 1           rcp85  1970 2100    60
## 2           rcp45  1970 2100    60
## 3           rcp85  1970 2100    60
## 4           rcp85  1970 2100    60
## 5           rcp45  1970 2100    60
## 6           rcp85  1970 2100    60
\end{verbatim}

\hypertarget{create-indices-bias-correct-projections}{%
\section{Create Indices \& bias correct
projections}\label{create-indices-bias-correct-projections}}

The next step creates indices based on the level 3 data for each hind
cast, historical run, and CMIP6 projection. The script below then bias
corrects each index using the historical run and recenters the
projection on the corresponding hindcast (such that projections are
\(\Delta\) from historical mean values for the reference period
\texttt{deltayrs\ \ \ \ \ \textless{}-\ 1970:2000} ).

\hypertarget{water-column-averaged-values}{%
\subsection{Water column averaged
values}\label{water-column-averaged-values}}

The average water column values for each variable from the ROMSNPZ model
strata x weekly Level2 outputs (`ACLIMregion') was calculated and used
to calculate the strata-area weighted mean value for the NEBS and SEBS
weekly, monthly, seasonally, and annually. Similarly, for survey
replicated (`ACLIMsurveyrep') Level2 outputs the average water column
value for each variable at each station was calculated used to calculate
the strata-area weighted mean value for the NEBS and SEBS annually.
These indices were calculate for hindcast, historical, and projection
scenarios, and used to bias correct the projections. More information on
the methods for each can be found in the tabs below and the code
immediately following this section will re-generate the bias corrected
indices. All of the bias corrected outputs can be found in the
``Data/out/CMIP6'' folder.

\begin{figure}
\centering
\includegraphics[width=0.75\textwidth,height=\textheight]{Figs/biascorrected_temp2.png}
\caption{\emph{Raw (top row) and bias corrected (bottom row)bottom
temperature indices based on survey replicated Level3 outputs for the
SEBS}}
\end{figure}

\emph{Important!} Note that for projections the `mn\_val' represents raw
mean values, while `val\_bias-corrected' is the bias corrected mn\_val
(should be used instead of the raw values). In all cases, for variables
that are log-normally distributed (cannot be \textless{} 0), the
ln(mn\_val) were used to bias correct and were then back transformed to
non-log space after correction:

For normally distributed variables (\(Y\)):
\[{Y}^{fut'}_{t,k} =\bar{Y}^{hind}_{k,\bar{T}} +\left( \frac{\sigma^{hind}_{k,\bar{T}}}{\sigma^{hist}_{k,\bar{T}}}*({Y}^{fut}_{t,k}-\bar{Y}^{hist}_{k,\bar{T}})  \right )\]

where \(\bar{Y}^{fut'}_{y,k}\) is the bias corrected varable \(k\) value
for time-step \(t\) (e.g., year, month, or season),
\(\bar{Y}^{hind}_{k,\bar{T}}\) is the mean value of the variable \(k\)
during the reference period \(\bar{T}=[1980,2013]\) from the hindcast
model, \(\sigma^{hind}_{k,\bar{T}}\) is the standard deviation of the
hindcast during the reference period \(\bar{T}\),
\(\sigma^{hist}_{k,\bar{T}}\) is the standard deviation of the
historical run during tje reference period, \({Y}^{fut}_{t,k}\) is the
value of the variable from the projection at time-step \(t\) and
\(\bar{Y}^{hist}_{k,\bar{T}}\) is the average value from the historical
run during reference period \(\bar{T}\).

For log-normally distributed variables(\(Y\)):
\[{Y}^{fut'}_{y,k} =e^{\ln\bar{Y}^{hind}_{k,\bar{T}} +\left( \frac{\hat{\sigma}^{hind}_{k,\bar{T}}}{\hat{\sigma}^{hist}_{k,\bar{T}}}*(\ln{Y}^{fut}_{t,k}-\ln\bar{Y}^{hist}_{k,\bar{T}})  \right )}\],
where \(\hat\sigma^{hist}_{k,\bar{T}}\) and
\(\hat\sigma^{hind}_{k,\bar{T}}\) are the standard deviation of the
\(\ln\bar{Y}^{hist}_{k,t}\) and \(\ln\bar{Y}^{hind}_{k,t}\) during the
reference period \(\hat{T}\) (respectively).

\hypertarget{weekly-indices}{%
\subsubsection{Weekly indices}\label{weekly-indices}}

Uses the strata x weekly data (`ACLIMregion') to generate
strata-specific averages in order to generate the strata area-weighted
averages for each week \(w\) each year \(y\).

\[\bar{Y}_{w,y,k}= \frac{\sum^{n_s}_{l}(\frac{1}{n_i}\sum^{n_t}_{t}Y_{k,w,y,s,t})*A_s} {\sum^{n_s}_{s}{A_s}}\],
where \(Y_{k,w,y,s,t}\) is the value of the variable \(k\) in strata
\(s\) at time \(t\) in year \(y\), \(A_s\) is the area of strata \(s\),
\(n_i\) is the number of stations in strata \(s\), and \(n_s\) is the
number of strata \(s\) in each basin (NEBS or SEBS).

\(\bar{Y}_{w,y,k}\) was calculated for the hindcast, historical run, and
projection time-series. For projections \(\bar{Y}_{w,y,k}\) was bias
corrected using the corresponding historical and hindcast values such
that:

\[\bar{Y}^{fut'}_{w,y,k} =\bar{Y}^{hind}_{w,k} +\left( \frac{\sigma^{hind}_{w,k}}{\sigma^{hist}_{w,k}}*(\bar{Y}^{fut}_{w,y,k}-\bar{Y}^{hist}_{w,k})  \right )\],
where \(\bar{Y}^{hist}_{w,k}\) and \(\bar{Y}^{hind}_{w,k}\) are the
average historical weekly values across years in the period (1980 to
2012 ; adjustable in \texttt{R/setup.R}).

\hypertarget{monthly-indices}{%
\subsubsection{Monthly indices}\label{monthly-indices}}

Uses the strata x weekly data (`ACLIMregion') to generate
strata-specific averages in order to generate the strata area-weighted
averages for each month \(m\) each year \(y\).

\[\bar{Y}_{m,y,k}= \frac{1}{n_w}\sum^{n_w}_{w}\bar{Y}_{w,y,k}\], where
\(\bar{Y}_{w,y,k}\) are the weekly average indices for variable \(k\) in
year \(y\) from the previous step ,\(n_w\) is the number of weeks in
each month \(m\).

\(\bar{Y}_{m,y,k}\) was calculated for the hindcast, historical run, and
projection time-series. For projections \(\bar{Y}_{m,y,k}\) was bias
corrected using the corresponding historical and hindcast values such
that:

\[\bar{Y}^{fut'}_{m,y,k} =\bar{Y}^{hind}_{m,k} +\left( \frac{\sigma^{hind}_{m,k}}{\sigma^{hist}_{m,k}}*(\bar{Y}^{fut}_{m,y,k}-\bar{Y}^{hist}_{m,k})  \right )\],
where \(\bar{Y}^{hist}_{m,k}\) and \(\bar{Y}^{hind}_{m,k}\) are the
average historical monthly values across years in the period (1980 to
2012 ; adjustable in \texttt{R/setup.R}).

\hypertarget{seasonal-indices}{%
\subsubsection{Seasonal indices}\label{seasonal-indices}}

Uses the strata x weekly data (`ACLIMregion') to generate
strata-specific averages in order to generate the strata area-weighted
averages for each season \(l\) each year \(y\).

\[\bar{Y}_{l,y,k}= \frac{1}{n_w}\sum^{n_w}_{w}\bar{Y}_{w,y,k}\], where
\(\bar{Y}_{w,y,k}\) are the weekly average indices for variable \(k\) in
year \(y\) from the previous step ,\(n_w\) is the number of weeks in
each season \(l\).

\(\bar{Y}_{l,y,k}\) was calculated for the hindcast, historical run, and
projection time-series. For projections \(\bar{Y}_{l,y,k}\) was bias
corrected using the corresponding historical and hindcast values such
that:

\[\bar{Y}^{fut'}_{l,y,k} =\bar{Y}^{hind}_{l,k} +\left( \frac{\sigma^{hind}_{l,k}}{\sigma^{hist}_{l,k}}*(\bar{Y}^{fut}_{l,y,k}-\bar{Y}^{hist}_{l,k})  \right )\],
where \(\bar{Y}^{hist}_{l,k}\) and \(\bar{Y}^{hind}_{l,k}\) are the
average historical seasonal values across years in the reference period
(1980 to 2012 ; adjustable in \texttt{R/setup.R}).

\hypertarget{annual-indices}{%
\subsubsection{Annual indices}\label{annual-indices}}

Uses the strata x weekly data (`ACLIMregion') to generate
strata-specific averages in order to generate the strata area-weighted
averages for each season \(l\) each year \(y\).

\[\bar{Y}_{y,k}= \frac{1}{n_w}\sum^{n_w}_{w}\bar{Y}_{w,y,k}\], where
\(\bar{Y}_{w,y,k}\) are the weekly average indices for variable \(k\) in
year \(y\) from the previous step ,\(n_w\) is the number of weeks in
each year \(y\).

\(\bar{Y}_{y,k}\) was calculated for the hindcast, historical run, and
projection time-series. For projections \(\bar{Y}_{y,k}\) was bias
corrected using the corresponding historical and hindcast values such
that:

\[\bar{Y}^{fut'}_{y,k} =\bar{Y}^{hind}_{k} +\left( \frac{\sigma^{hind}_{k}}{\sigma^{hist}_{k}}*(\bar{Y}^{fut}_{y,k}-\bar{Y}^{hist}_{k})  \right )\],
where \(\bar{Y}^{hind}_{k}\) and \(\bar{Y}^{hist}_{k}\) are the average
historical values across years in the reference period (1980 to 2012 ;
adjustable in \texttt{R/setup.R}).

\hypertarget{annual-survey-rep.-indices}{%
\subsubsection{Annual survey rep.
indices}\label{annual-survey-rep.-indices}}

Uses the station specific survey replicated (in time and space) data
(`ACLIMsurveyrep') to generate strata-specific averages in order to
generate the strata area-weighted averages for each year \(y\).

\[\bar{Y}_{y,k}= \frac{\sum^{n_s}_{l}(\frac{1}{n_i}\sum^{n_i}_{i}Y_{k,y,s,i})*A_s} {\sum^{n_s}_{s}{A_s}}\],
where \(Y_{k,y,s,i}\) is the value of the variable \(k\) at station
\(i\) in strata \(s\) in year \(y\), \(A_s\) is the area of strata
\(s\), \(n_i\) is the number of stations in strata \(s\), and \(n_s\) is
the number of strata \(s\) in each basin (NEBS or SEBS).

\(\bar{Y}_{y,k}\) was calculated for the hindcast, historical run, and
projection time-series. For projections \(\bar{Y}_{y,k}\) was bias
corrected using the corresponding historical and hindcast values such
that:

\[\bar{Y}^{fut'}_{y,k} =\bar{Y}^{hind}_{k} +\left( \frac{\sigma^{hind}_{k}}{\sigma^{hist}_{k}}*(\bar{Y}^{fut}_{y,k}-\bar{Y}^{hist}_{k})  \right )\],
where \(\bar{Y}^{hind}_{k}\) and \(\bar{Y}^{hist}_{k}\) are the average
historical values across years in the reference period (1980 to 2012 ;
adjustable in \texttt{R/setup.R}).

Appendix A includes the code used to generate the ACLIM2 indices and
bias correct them. That code can be run to re-make the indices if you
like but takes approx 30 mins a CMIP to run.

\hypertarget{explore-aclim2-indices}{%
\section{Explore ACLIM2 Indices}\label{explore-aclim2-indices}}

The following code will open an interactive shiny() app for exploring
the indices. You can also view this online at
(kkh2022.shinyapps.io/ACLIM2\_indices){[}\url{https://kkh2022.shinyapps.io/ACLIM2_indices/}{]}.

\begin{Shaded}
\begin{Highlighting}[]
\NormalTok{shiny}\SpecialCharTok{::}\FunctionTok{runApp}\NormalTok{(}\StringTok{"/Users/kholsman/Documents/GitHub/ACLIM2/R/shiny\_aclim/ACLIM2\_indices/app.R"}\NormalTok{)}
\end{Highlighting}
\end{Shaded}

\begin{figure}
\centering
\includegraphics[width=1\textwidth,height=\textheight]{Figs/biascorrected_temp2.png}
\caption{``Raw (top row) and bias corrected (bottom row)bottom
temperature indices based on survey replicated Level3 outputs for the
SEBS''}
\end{figure}

\hypertarget{annual-indices-1}{%
\subsection{Annual indices}\label{annual-indices-1}}

\begin{Shaded}
\begin{Highlighting}[]
    \CommentTok{\# {-}{-}{-}{-}{-}{-}{-}{-}{-}{-}{-}{-}{-}{-}{-}{-}{-}{-}{-}{-}{-}{-}{-}{-}{-}{-}{-}{-}{-}{-}{-}{-}{-}{-}{-}{-}{-}{-}}
    \CommentTok{\# SETUP WORKSPACE}
\NormalTok{    main   }\OtherTok{\textless{}{-}} \FunctionTok{getwd}\NormalTok{()  }\CommentTok{\#"\textasciitilde{}/GitHub\_new/ACLIM2"}
    
    \CommentTok{\# loads packages, data, setup, etc.}
    \FunctionTok{suppressMessages}\NormalTok{(}\FunctionTok{source}\NormalTok{(}\StringTok{"R/make.R"}\NormalTok{))}
    
    
    \CommentTok{\# load the Indices:}
\NormalTok{    fldr }\OtherTok{\textless{}{-}} \StringTok{"Data/out/K20P19\_CMIP6/allEBS\_means"}
\NormalTok{    dirlist }\OtherTok{\textless{}{-}}\FunctionTok{grep}\NormalTok{ (}\StringTok{"annual"}\NormalTok{, }\FunctionTok{dir}\NormalTok{(fldr))}
    \ControlFlowTok{for}\NormalTok{(d }\ControlFlowTok{in}\NormalTok{ dirlist)}
      \FunctionTok{load}\NormalTok{(}\FunctionTok{file.path}\NormalTok{(fldr,d))}
\NormalTok{    hnd }\OtherTok{\textless{}{-}}\NormalTok{ ACLIM\_annual\_fut\_mn}
    
\StringTok{\textasciigrave{}\textasciigrave{}\textasciigrave{}\textasciigrave{}}

\DocumentationTok{\#\# Monthly indices}

\DocumentationTok{\#\# Seasonal indices}

\DocumentationTok{\#\# weekly indices}

\CommentTok{\# Output to .dat file (ADMB/ TMB users)}

\CommentTok{\# Special cases \{.tabset\}}

\DocumentationTok{\#\# monthly indices (Andy)}
\end{Highlighting}
\end{Shaded}

\hypertarget{nrs-indices-andruxe9}{%
\subsection{NRS indices (André)}\label{nrs-indices-andruxe9}}

\hypertarget{salmon-index-ellen}{%
\subsection{Salmon index (Ellen)}\label{salmon-index-ellen}}

\hypertarget{nfs-index-jeremy}{%
\subsection{NFS index (Jeremy )}\label{nfs-index-jeremy}}

\hypertarget{appendix-a-create-indices-and-bias-correct-cmip6-projections}{%
\section{APPENDIX A: Create indices and bias correct CMIP6
projections}\label{appendix-a-create-indices-and-bias-correct-cmip6-projections}}

The following code shows how the ACLIM2 indices and bias correction was
done. You do not need to re-run this (it is included so you can if you
want to). To explore the indices skep to the next section.

\begin{Shaded}
\begin{Highlighting}[]
    \CommentTok{\# {-}{-}{-}{-}{-}{-}{-}{-}{-}{-}{-}{-}{-}{-}{-}{-}{-}{-}{-}{-}{-}{-}{-}{-}{-}{-}{-}{-}{-}{-}{-}{-}{-}{-}{-}{-}{-}{-}}
    \CommentTok{\# SETUP WORKSPACE}
    \CommentTok{\# setwd("Documents/GitHub/ACLIM2")}
\NormalTok{    tmstp  }\OtherTok{\textless{}{-}} \FunctionTok{format}\NormalTok{(}\FunctionTok{Sys.time}\NormalTok{(), }\StringTok{"\%Y\_\%m\_\%d"}\NormalTok{)}
\NormalTok{    main   }\OtherTok{\textless{}{-}} \FunctionTok{getwd}\NormalTok{()  }\CommentTok{\#"\textasciitilde{}/GitHub\_new/ACLIM2"}
    
    \CommentTok{\# loads packages, data, setup, etc.}
    \FunctionTok{suppressMessages}\NormalTok{(}\FunctionTok{source}\NormalTok{(}\StringTok{"R/make.R"}\NormalTok{))}
    
\NormalTok{    tmstamp1  }\OtherTok{\textless{}{-}} \FunctionTok{format}\NormalTok{(}\FunctionTok{Sys.time}\NormalTok{(), }\StringTok{"\%Y\%m\%d"}\NormalTok{)}
    \CommentTok{\# tmstamp1  \textless{}{-} "20220428"}
    
\NormalTok{    update\_hind  }\OtherTok{\textless{}{-}} \ConstantTok{TRUE}   \CommentTok{\# set to true to update hind and hindS; needed annually}
\NormalTok{    update\_proj  }\OtherTok{\textless{}{-}} \ConstantTok{TRUE}   \CommentTok{\# set to true to update fut; not needed}
\NormalTok{    update\_hist  }\OtherTok{\textless{}{-}} \ConstantTok{TRUE}   \CommentTok{\# set to true to update fut; not needed}
     
    \CommentTok{\# the reference years for bias correcting in R/setup.R}
\NormalTok{    ref\_years }
    
    \CommentTok{\# the year to z{-}score scale / delta in R/setup.R}
\NormalTok{    deltayrs }
    
    \CommentTok{\# remove these variables:}
\NormalTok{    vl1 }\OtherTok{\textless{}{-}}\NormalTok{ srvy\_vars[}\SpecialCharTok{!}\NormalTok{srvy\_vars}\SpecialCharTok{\%in\%}\FunctionTok{c}\NormalTok{(}\StringTok{"station\_id"}\NormalTok{,}\StringTok{"latitude"}\NormalTok{,}
                                    \StringTok{"longitude"}\NormalTok{,}\StringTok{"stratum"}\NormalTok{,}\StringTok{"doy"}\NormalTok{,}
                                  \StringTok{"Iron\_bottom5m"}\NormalTok{,}\StringTok{"Iron\_integrated"}\NormalTok{,}
                                  \StringTok{"Iron\_surface5m"}\NormalTok{,}\StringTok{"prod\_Eup\_integrated"}\NormalTok{,}
                                  \StringTok{"prod\_NCa\_integrated"}\NormalTok{)]}
\NormalTok{    vl2 }\OtherTok{\textless{}{-}}\NormalTok{ weekly\_vars[}\SpecialCharTok{!}\NormalTok{weekly\_vars}\SpecialCharTok{\%in\%}\FunctionTok{c}\NormalTok{(}\StringTok{"station\_id"}\NormalTok{,}\StringTok{"latitude"}\NormalTok{,}
                                    \StringTok{"longitude"}\NormalTok{,}\StringTok{"stratum"}\NormalTok{,}\StringTok{"doy"}\NormalTok{,}
                                  \StringTok{"Iron\_bottom5m"}\NormalTok{,}\StringTok{"Iron\_integrated"}\NormalTok{,}
                                  \StringTok{"Iron\_surface5m"}\NormalTok{,}\StringTok{"prod\_Eup\_integrated"}\NormalTok{,}
                                  \StringTok{"prod\_NCa\_integrated"}\NormalTok{)]}
\NormalTok{    vl}\OtherTok{\textless{}{-}}\FunctionTok{unique}\NormalTok{(}\FunctionTok{c}\NormalTok{(vl1,vl2))}
    \CommentTok{\# add in largeZoop (gets generated in make\_indices\_region\_new.R)}
\NormalTok{    vl }\OtherTok{\textless{}{-}} \FunctionTok{c}\NormalTok{(vl,}\StringTok{"largeZoop\_integrated"}\NormalTok{)}

    \CommentTok{\# Identify which variables would be normally }
    \CommentTok{\# distributed (i.e., can have negative values)}
\NormalTok{     normvl }\OtherTok{\textless{}{-}} \FunctionTok{c}\NormalTok{(}\StringTok{"shflux"}\NormalTok{,}\StringTok{"ssflux"}\NormalTok{,}\StringTok{"temp\_bottom5m"}\NormalTok{,}
      \StringTok{"temp\_integrated"}\NormalTok{,}\StringTok{"temp\_surface5m"}\NormalTok{,}
      \StringTok{"uEast\_bottom5m"}\NormalTok{,}\StringTok{"uEast\_surface5m"}\NormalTok{,}
      \StringTok{"vNorth\_bottom5m"}\NormalTok{,}\StringTok{"vNorth\_surface5m"}\NormalTok{)}

\NormalTok{    normlist }\OtherTok{\textless{}{-}} \FunctionTok{data.frame}\NormalTok{(}\AttributeTok{var =}\NormalTok{ vl, }\AttributeTok{lognorm =} \ConstantTok{TRUE}\NormalTok{)}
\NormalTok{    normlist}\SpecialCharTok{$}\NormalTok{lognorm[normlist}\SpecialCharTok{$}\NormalTok{var}\SpecialCharTok{\%in\%}\NormalTok{normvl] }\OtherTok{\textless{}{-}} \ConstantTok{FALSE}

    
    \CommentTok{\# generate indices and bias corrected projections }
    \CommentTok{\# This takes approx 30 mins each}
    
\NormalTok{    gcmcmipL }\OtherTok{\textless{}{-}} \FunctionTok{c}\NormalTok{(}\StringTok{"B10K{-}K20P19\_CMIP6\_miroc"}\NormalTok{,}
                  \StringTok{"B10K{-}K20P19\_CMIP6\_gfdl"}\NormalTok{,}
                  \StringTok{"B10K{-}K20P19\_CMIP6\_cesm"}\NormalTok{) }
\NormalTok{    CMIP6\_Indices }\OtherTok{\textless{}{-}} \FunctionTok{suppressMessages}\NormalTok{(}
                        \FunctionTok{makeACLIM2\_Indices}\NormalTok{(}
                        \AttributeTok{BC\_target =} \StringTok{"mn\_val"}\NormalTok{,}
                        \AttributeTok{hind\_sim  =}  \StringTok{"B10K{-}K20\_CORECFS"}\NormalTok{,}
                        \AttributeTok{histLIST  =} \FunctionTok{paste0}\NormalTok{(gcmcmipL,}\StringTok{"\_historical"}\NormalTok{),}
                        \AttributeTok{gcmcmipLIST =}\NormalTok{ gcmcmipL,}
                        \AttributeTok{sim\_listIN =}\NormalTok{ sim\_list[}\SpecialCharTok{{-}}\FunctionTok{grep}\NormalTok{(}\StringTok{"historical"}\NormalTok{,sim\_list)]))}
    
     \ControlFlowTok{if}\NormalTok{(}\StringTok{"CMIP6\_Indices"}\SpecialCharTok{\%in\%}\FunctionTok{ls}\NormalTok{())\{                 }
      \FunctionTok{save\_indices}\NormalTok{(}\AttributeTok{flIN =}\NormalTok{ CMIP6\_Indices, }
                   \AttributeTok{subfl =} \StringTok{"allEBS\_means"}\NormalTok{,}
                   \AttributeTok{post\_txt =} \StringTok{"\_mn"}\NormalTok{,}
                   \AttributeTok{CMIP\_fdlr =}\StringTok{"K20P19\_CMIP6"}\NormalTok{)}
\NormalTok{      fl }\OtherTok{\textless{}{-}} \StringTok{"Data/out/CMIP6\_Indices\_List.Rdata"}
      
      \ControlFlowTok{if}\NormalTok{(}\FunctionTok{file.exists}\NormalTok{(fl)) }\FunctionTok{file.remove}\NormalTok{(fl)}
      \FunctionTok{save}\NormalTok{(CMIP6\_Indices, }\AttributeTok{file =}\NormalTok{ fl)}
      \FunctionTok{rm}\NormalTok{(CMIP6\_Indices)}
      \FunctionTok{gc}\NormalTok{()}
\NormalTok{     \}}
    \CommentTok{\# CMIP5 K20P19}
\NormalTok{    gcmcmipL2 }\OtherTok{\textless{}{-}} \FunctionTok{c}\NormalTok{(}\StringTok{"B10K{-}K20P19\_CMIP5\_MIROC"}\NormalTok{,}\StringTok{"B10K{-}K20P19\_CMIP5\_GFDL"}\NormalTok{,}\StringTok{"B10K{-}K20P19\_CMIP5\_CESM"}\NormalTok{) }
\NormalTok{    CMIP5\_K20\_Indices }\OtherTok{\textless{}{-}} \FunctionTok{suppressMessages}\NormalTok{(}
                        \FunctionTok{makeACLIM2\_Indices}\NormalTok{(}
                        \AttributeTok{BC\_target =} \StringTok{"mn\_val"}\NormalTok{,}
                        \AttributeTok{hind\_sim  =}  \StringTok{"B10K{-}K20\_CORECFS"}\NormalTok{,}
                        \AttributeTok{histLIST  =} \FunctionTok{paste0}\NormalTok{(gcmcmipL,}\StringTok{"\_historical"}\NormalTok{),}
                        \AttributeTok{gcmcmipLIST =}\NormalTok{ gcmcmipL2,}
                        \AttributeTok{sim\_listIN =}\NormalTok{ sim\_list[}\SpecialCharTok{{-}}\FunctionTok{grep}\NormalTok{(}\StringTok{"historical"}\NormalTok{,sim\_list)]))}
    
    \ControlFlowTok{if}\NormalTok{(}\StringTok{"CMIP5\_K20\_Indices"}\SpecialCharTok{\%in\%}\FunctionTok{ls}\NormalTok{())\{}
        \FunctionTok{save\_indices}\NormalTok{(}\AttributeTok{flIN =}\NormalTok{ CMIP5\_K20\_Indices, }
                     \AttributeTok{subfl =} \StringTok{"allEBS\_means"}\NormalTok{,}
                     \AttributeTok{post\_txt =} \StringTok{"\_mn"}\NormalTok{,}
                     \AttributeTok{CMIP\_fdlr =}\StringTok{"K20P19\_CMIP5"}\NormalTok{)}
        
\NormalTok{        fl }\OtherTok{\textless{}{-}} \StringTok{"Data/out/CMIP5\_K20\_Indices\_List.Rdata"}
        \ControlFlowTok{if}\NormalTok{(}\FunctionTok{file.exists}\NormalTok{(fl)) }\FunctionTok{file.remove}\NormalTok{(fl)}
        \FunctionTok{save}\NormalTok{(CMIP5\_K20\_Indices, }\AttributeTok{file =}\NormalTok{ fl)}
        \FunctionTok{rm}\NormalTok{(CMIP5\_K20\_Indices)}
        \FunctionTok{gc}\NormalTok{()}
\NormalTok{    \}}
    \CommentTok{\# CMIP5 H16}
\NormalTok{    gcmcmipL2 }\OtherTok{\textless{}{-}} \FunctionTok{c}\NormalTok{(}\StringTok{"B10K{-}H16\_CMIP5\_MIROC"}\NormalTok{,}\StringTok{"B10K{-}H16\_CMIP5\_GFDL"}\NormalTok{,}\StringTok{"B10K{-}H16\_CMIP5\_CESM"}\NormalTok{) }
\NormalTok{    CMIP5\_H16\_Indices }\OtherTok{\textless{}{-}} \FunctionTok{suppressMessages}\NormalTok{(}
                        \FunctionTok{makeACLIM2\_Indices}\NormalTok{(}
                        \AttributeTok{BC\_target =} \StringTok{"mn\_val"}\NormalTok{,}
                        \AttributeTok{hind\_sim  =}  \StringTok{"B10K{-}H16\_CORECFS"}\NormalTok{,}
                        \AttributeTok{histLIST  =} \FunctionTok{rep}\NormalTok{(}\StringTok{"B10K{-}H16\_CORECFS"}\NormalTok{,}\DecValTok{3}\NormalTok{),}
                        \AttributeTok{gcmcmipLIST =}\NormalTok{ gcmcmipL2,}
                        \AttributeTok{sim\_listIN =}\NormalTok{ sim\_list[}\SpecialCharTok{{-}}\FunctionTok{grep}\NormalTok{(}\StringTok{"historical"}\NormalTok{,sim\_list)]))}
    \ControlFlowTok{if}\NormalTok{(}\StringTok{"CMIP5\_H16\_Indices"}\SpecialCharTok{\%in\%}\FunctionTok{ls}\NormalTok{())\{}
      \FunctionTok{save\_indices}\NormalTok{(}\AttributeTok{flIN =}\NormalTok{ CMIP5\_H16\_Indices, }
                   \AttributeTok{subfl =} \StringTok{"allEBS\_means"}\NormalTok{,}
                   \AttributeTok{post\_txt =} \StringTok{"\_mn"}\NormalTok{,}
                   \AttributeTok{CMIP\_fdlr =}\StringTok{"H16\_CMIP5"}\NormalTok{)}
     
\NormalTok{      fl }\OtherTok{\textless{}{-}} \StringTok{"Data/out/CMIP5\_H16\_Indices\_List.Rdata"}
      \ControlFlowTok{if}\NormalTok{(}\FunctionTok{file.exists}\NormalTok{(fl)) }\FunctionTok{file.remove}\NormalTok{(fl)}
      \FunctionTok{save}\NormalTok{(CMIP5\_H16\_Indices, }\AttributeTok{file =}\NormalTok{ fl)}
      \FunctionTok{rm}\NormalTok{(CMIP5\_H16\_Indices)}
      \FunctionTok{gc}\NormalTok{()}
\NormalTok{    \}}
    \ControlFlowTok{if}\NormalTok{(}\DecValTok{1}\SpecialCharTok{==}\DecValTok{10}\NormalTok{)\{}
      \FunctionTok{save}\NormalTok{(CMIP6\_Indices, }\AttributeTok{file =} \StringTok{"Data/out/CMIP6\_Indices\_List.Rdata"}\NormalTok{)}
      \FunctionTok{save}\NormalTok{(CMIP5\_K20\_Indices, }\AttributeTok{file =} \StringTok{"Data/out/CMIP5\_K20\_Indices\_List.Rdata"}\NormalTok{)}
      \FunctionTok{save}\NormalTok{(CMIP5\_H16\_Indices, }\AttributeTok{file =} \StringTok{"Data/out/CMIP5\_H16\_Indices\_List.Rdata"}\NormalTok{)}
\NormalTok{    \}}
\end{Highlighting}
\end{Shaded}

\begin{figure}
\centering
\includegraphics[width=0.75\textwidth,height=\textheight]{Figs/Hind_Sept_large_Zoop.jpg}
\caption{September large zooplankton integrated concentration}
\end{figure}

\hypertarget{misc}{%
\section{misc}\label{misc}}

\[B0^k_{input}= \bar{B0}^k_{(2004:2014)}\left(\frac{B0^{a}_{2015}}{\bar{B0}^a_{(2004:2014)}}\right) \]
Where B0kinput is the unfished biomass used for setting inputs of (e.g.,
B0ktarget = 0.4B0kinput) and is determined by re-scaling the spawning
stock biomass from the status quo assessment in 2015 (B0a2015) to the
average model spawning stock biomass for your model between 2004-2014
(i.e., B0k) using the average unfished biomass from the stock assessment
model during the same period (B0a).

\end{document}
